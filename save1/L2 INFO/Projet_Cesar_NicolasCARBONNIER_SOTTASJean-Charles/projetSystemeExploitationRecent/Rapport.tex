\documentclass[a4]{article}


\usepackage[utf8]{inputenc}
\usepackage[french]{babel}


\author{CARBONNIER Nicolas, SOTTAS Jean-Charles }
\title{Rapport du projet Systèmes d'Exploitation Avé Cesar}
\date{\today}


\begin{document}
\maketitle
\section{Description du programme}
\subsection{Structure INFO}
La structure sauvegarde le nombre de lignes dans la variable taille, et sauvegarde chaque ligne dans le tableau ligne.
\subsection{Structure ENCODAGE}
La structure sauvegarde un caractère, le numéro de décalage qui doit être fait et le thread qui emploie cette structure.
\subsection{Processus Directeur}
Le processus directeur crée deux tubes, un pour que le père puisse écrire et le fils lire et l'autre pour que le fils puisse écrire et que le père lise, pour chaque processus fils créé. Le père va ainsi envoyer une ligne du fichier principal dans chaque tube. Il va recevoir dans l'autre tube soit le message en clair, soit "C" correspondant au fichier qui a été crypté. 
\subsection{Processus Chef-Equipe}
Le processus chef d'équipe découpe le message re\c cu dans le tube, et obtiendra le nom du fichier avec son chemin, soit c pour crypter soit d pour décrypter. Il ouvre le fichier et commence à envoyer aux threads un caractère; il enverra aux threads décrypter sinon encrypter. Il stockera leurs retours dans un buffer qui sera ensuite écrit dans un fichier ou envoyé dans le tube pour le processus directeur.
\subsection{Processus Fils}
Le processus fils re\c coit un caractère et le numéro de décalage. Il vérifie si le caractère est une lettre; si c'est une lettre il vérifie si celle-ci est une majuscule, puis lui applique la formule. Si celle-ci n'est pas une lettre alors le caractère est retourné.
\subsection{Ecrire fichier}
La fonction re\c coit le buffer ainsi que le nom du fichier avec son chemin. La fonction va alors chercher à modifier le nom du fichier pour permettre le rajout de l'extention "crypter". Une fois ajouté on crée le fichier avec les mêmes droits et la même destination puis on écrit le message codé.
\section{Les difficultés}
Pour le tube qui retourne au directeur, la difficulté a été le choix du message de retour lorsque le retour doit être vide : il est impossible de lire dans un tube vide.\newline 
\end{document}
